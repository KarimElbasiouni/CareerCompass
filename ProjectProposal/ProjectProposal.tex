\documentclass[12pt]{article}
\usepackage[margin=0.8in]{geometry}      % 0.8" margins
\usepackage{setspace}
\onehalfspacing                          % 1.5 line spacing
\usepackage{newtxtext,newtxmath}        % Times-like font (Times New Roman equivalent)
\usepackage{hyperref}
\usepackage{array}
\usepackage{booktabs}

\begin{document}

\begin{center}
  {\Large\bfseries Project Title: CareerCompass}\\[8pt]
  {\normalsize Group 85}
\end{center}

\section*{1. Team members}
\begin{itemize}
  \item Member 1: Name  — email
  \item Member 2: Name  — email
  \item Member 3: Name  — email
\end{itemize}

\section*{2. Task title and overview}
Task title and overview, including the significance and what makes it challenging 

\section*{3. Task definition}
3. Task definition (type of data, classification/regression/generation, number of classes, single label or multi-label) 
\begin{itemize}
  \item Data type(s): (e.g., text, images, audio, tabular, multimodal).
  \item Task category: classification / regression / generation / detection / segmentation / ranking.
  \item Number of classes (if classification): e.g., 3 classes: \{A, B, C\}
  \item Labeling scheme: single-label or multi-label. If multi-label, list the labels and whether they are independent.
  \item Input / output shapes (short): e.g., input: text (<= 256 tokens), output: single class label.
\end{itemize}

\section*{4. Problem description, impact, and challenges}
4. Describe the problem, it’s impact on the real world, and why it is challenging \\ \\
\textbf{Problem description:} One concise paragraph explaining the problem and the stakeholders affected.\\[4pt]
\textbf{Real-world impact:} Bulleted list (e.g., business benefit, societal impact, cost/time savings).\\[4pt]
\textbf{Why it's challenging:}
\begin{itemize}
  \item Data-specific issues (noise, missing values, unbalanced classes).
  \item Task-specific issues (ambiguous labels, domain shift, fine-grained distinctions).
  \item Operational issues (rate limits, privacy/regulation, annotation cost).
\end{itemize}

\section*{5. Data sources and collection plan}
 Data source(s) and plan for data collection. This may include how you are going to scrape the data
  and follow terms-of-service, API access and handling rate limiting, using open-source data, or 
  any other relevant details. If your data does not have labels, how do you plan to get them? If 
  assigning labels by hand, how long does it take per instance? Include links to the data if relevant.
  If you download your dataset from Kaggle, you must include the Kaggle link and the original data source link
  from where the dataset was downloaded and posted on Kaggle. Include any meta-data available for 
  your corpus. If you are not collecting it yourself, include details about how the data was collected, annotated, preprocessed, etc. Do not work on the
 datasets for which no source information is available. Please indicate whether you will use a small
 subset of the data or features for your project or the entire dataset, and why. 

\begin{itemize}
  \item Primary data source(s): name the dataset(s) and provide direct links. If using Kaggle, include both the Kaggle page and the dataset's original source.
    \begin{itemize}
      \item Example: \href{https://www.kaggle.com/example}{Kaggle: Example Dataset} — original source: \href{https://original-source.example}{Original Source}
    \end{itemize}
  \item If scraping / API: explain planned approach and compliance
    \begin{itemize}
      \item API endpoints or websites to query.
      \item Authentication and rate-limiting strategy (API keys, exponential backoff, pagination).
      \item Respecting terms of service and robots.txt. Provide a short sentence about legal/ethical compliance.
    \end{itemize}
  \item Labeling plan:
    \begin{itemize}
      \item If labels exist: note format and any preprocessing needed.
      \item If labeling manually or via crowdsourcing: describe annotation guidelines, estimated time per instance, and consensus process (e.g., 3 annotators + majority vote). Example: 90 seconds per instance $\Rightarrow$ 1000 examples $\approx$ 25 hours of annotation.
    \end{itemize}
  \item Preprocessing:
    \begin{itemize}
      \item Cleaning steps (deduplication, normalization, resizing, tokenization).
      \item Handling missing data and noisy labels.
    \end{itemize}
  \item Subsetting decision:
    \begin{itemize}
      \item Will you use entire dataset or a subset? State reason (compute constraints, label quality, class balance).
    \end{itemize}
  \item Metadata available: list fields like timestamps, source, author, confidence scores, etc.
\end{itemize}

\section*{6. Expected dataset size and example datapoints}
6. Expected size of the dataset (number of data points) and 3 example data points with labels. Your dataset should have at least 1k data points. 
Some projects will have more or less data.\\ 
\textbf{Expected size:} Approximate number of data points (must be $\ge$ 1,000). Example: 10,000 instances.\\[6pt]
\textbf{Three example data points (show input $\rightarrow$ label):} \\
\begin{tabular}{p{0.62\linewidth} p{0.32\linewidth}}
\toprule
Example input & Label(s) \\
\midrule
"Example text or short description of instance 1" & Class A \\
"Example text or short description of instance 2" & Class B \\
"Example text or short description of instance 3" & Class A, Class C (if multi-label) \\
\bottomrule
\end{tabular}

\section*{7. Proposed solution}
7. Proposed solution: How do you plan to go about solving this problem? It is okay to not know how the machine learning models work at this point in the class,
but you should be starting to get some idea based on the lectures and assigned readings. What kind of features and target labels do you have? What kind of models 
might you try? You may not propose simple linear regression models for this project. Are there any existing solutions to your problem? You must indicate 2-5 sources 
(research papers, books, machine learning challenges online) from where your solution was inspired. Put some thought into how you would approach this. How will you 
know if the model is good? How will you evaluate it? Also, share the libraries you intend to use for the project.  
\textbf{High-level approach:} Short paragraph describing end-to-end plan (data $\rightarrow$ features $\rightarrow$ model $\rightarrow$ evaluation).\\[4pt]

\textbf{Features / inputs:}
\begin{itemize}
  \item Raw features (e.g., text, images).
  \item Engineered features (e.g., TF-IDF, sentiment, metadata, image embeddings).
  \item Data augmentation strategies where appropriate.
\end{itemize}

\textbf{Model candidates:}
\begin{itemize}
  \item Baseline(s): simple non-linear baselines such as Random Forest, XGBoost, or a small CNN/classifier trained from scratch.
  \item Advanced models: pre-trained transformer (e.g., BERT/RoBERTa) fine-tuning; or pre-trained ResNet/ViT for images; or multimodal fusion approaches if applicable.
  \item Avoid only linear regression; include at least one deep or ensemble method.
\end{itemize}

\textbf{Evaluation plan:}
\begin{itemize}
  \item Metrics: accuracy / F1 (micro \& macro) / precision \& recall / AUC-ROC / RMSE (if regression). Choose metrics matching task and class imbalance.
  \item Validation scheme: train / validation / test split, possibly stratified k-fold cross-validation.
  \item Baseline vs target performance: define success criteria (e.g., beat baseline by X\% absolute F1 or achieve F1 $\ge$ 0.75).
  \item Error analysis \& robustness checks: confusion matrix, failure case examples, calibration, domain shift testing.
\end{itemize}

\textbf{Libraries / tools:}
\begin{itemize}
  \item Data handling: Python, pandas, numpy
  \item Modelling: PyTorch or TensorFlow / Hugging Face Transformers / scikit-learn / XGBoost
  \item NLP tools (if text): spaCy, NLTK, Hugging Face Datasets
  \item Web collection: requests, BeautifulSoup, Selenium (if necessary) or official APIs / tweepy
  \item Experiment tracking: Weights \& Biases / MLflow (optional)
\end{itemize}

\textbf{Related work / references (2--5 sources):}


\end{document}