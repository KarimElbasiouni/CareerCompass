\documentclass[12pt]{article}
\usepackage[margin=0.8in]{geometry}      % 0.8" margins
\usepackage{setspace}
\onehalfspacing                          % 1.5 line spacing
\usepackage{newtxtext,newtxmath}        % Times-like font (Times New Roman equivalent)
\usepackage{hyperref}
\usepackage{array}
\usepackage{booktabs}

\begin{document}

\begin{center}
  {\Large\bfseries Project Title: CareerCompass}\\[8pt]
  {\normalsize Group 85}
\end{center}

\section*{1. Team members}
\begin{itemize}
  \item Imran Chowdhury — 400470828 — chowdi13@mcmaster.ca
  \item Member 2: Karim Elbasiouni — email
  \item Member 3: Rami Abu Sultan — email
\end{itemize}

\section*{2. Task title and overview}
\textbf{Title:} CareerCompass — Resume $\rightarrow$ Job Title \& Occupation Family Classification

\textbf{Overview:} Build a system that ingests a resume (PDF/DOCX/TXT or pasted text) and predicts (1) a fine-grained job title (e.g., Software Engineer, Data Analyst) and (2) a coarse occupation family (O*NET/SOC major group, e.g., Computer \& Mathematical Occupations).  

\textbf{Significance:} Automated routing of candidates improves recruiter efficiency, speeds screening, and enables analytics on talent pools.  

\textbf{Challenges:} Resumes are unstructured; titles are noisy; class imbalance; PII must be removed; hierarchical consistency between title and family must be maintained.

\setlength{\parindent}{0pt}

\section*{3. Task definition}
\subsection*{3.1 Data Type (what the model sees)}
The model operates primarily on de-identified resume text consolidated from the Summary, Experience, Skills, and Education sections after converting PDF or DOCX files to text. It supplements this corpus with derived auxiliary features such as section presence flags, token counts, simple skill-dictionary hits (e.g., Python, SQL, AWS), and basic length statistics while removing names, emails, phone numbers, addresses, and other obvious contact tokens before any storage or logging.

\medskip

\subsection*{3.2 Observed Label Space}
Fine-grained titles span 36 distinct categories that include roles such as Java Developer, Data Science, DevOps, Business Analyst, Product Manager, Technical Writer, SRE, and Engineering Manager. Counts are skewed, with several titles represented by 150--200 examples while roughly a dozen classes have fewer than 60, so alias normalization is required to keep the space tractable.

\medskip

Coarse occupation families align the titles to SOC/O*NET major groups with realistic coverage of five to seven families in this corpus, primarily Computer \& Mathematical along with Management, Business/Operations, Media/Communications, and a short tail. The full SOC major-group set remains available, but training and evaluation focus on families present in the data to avoid empty classes.

\medskip

\subsection*{3.3 Learning Setup}
The model architecture follows a two-head single-label classification design that leverages a shared encoder.
\begin{enumerate}
  \item Head A: Job Title operates as a single-label, multi-class classifier targeting approximately 25--30 cleaned classes after alias-normalizing near duplicates (e.g., SWE, SDE II, Senior Java Developer mapped to Software Engineer) and collapsing ultra-rare labels with fewer than 40--60 samples into an Other or parent bucket to manage imbalance and evaluation noise.
  \item Head B: Occupation Family provides single-label, multi-class predictions across the five to seven observed families, combining direct supervision on family labels with a consistency constraint enforced through the title-to-family mapping table.
\end{enumerate}

\medskip

\subsection*{3.4 Model I/O}
\begin{enumerate}
  \item \textit{Input} describes normalized resume text alongside the minimal derived features that capture structural cues.
  \item \textit{Output} returns title probabilities over $K\approx25$--$30$ classes (Top-1 and Top-3) and family probabilities over $F = 5$--$7$ families (Top-1 with $\sim$5--$7$ families) together with rationale highlights such as ``Python, REST, AWS, CI/CD''.
  \item \textit{Interfaces} support batch CSV processing and a REST \texttt{/predict} endpoint that produces \{\texttt{title\_top1}, \texttt{title\_top3}, \texttt{family}, \texttt{confidences}, \texttt{reasons[]}\}.
\end{enumerate}

\medskip

\subsection*{3.5 Why Two Heads (and not one label)}
Predicting both title and family improves robustness because the fine-grained title space is noisy while the family space is coarser and more stable, enables hierarchical checks by flagging disagreements between the predicted title and family, and strengthens intern or new-grad routing where the family often drives track selection while the title provides the display string.

\medskip

\subsection*{3.6 Evaluation Protocol}
\begin{enumerate}
  \item Title head metrics report Top-1 and Top-3 accuracy alongside macro-F1 to respect long-tail classes and confusion reviews among neighboring titles.
  \item Family head metrics track accuracy and hierarchical correctness to credit predictions that land in the right family when title classification misses.
  \item Calibration uses the Brier score and reliability curves to determine confidence thresholds for triage.
  \item Splits follow stratified train, validation, and test partitions with a held-out slice emphasizing resumes that look ``real-style'' rather than synthetic to assess generalization.
  \item Ablations compare text-only versus text-plus-auxiliary inputs, SVM or GBDT baselines versus DistilBERT or SBERT encoders, and configurations with or without tail collapse.
\end{enumerate}

\medskip

\subsection*{3.7 Planned Models}
The initial baseline relies on TF-IDF uni- and bi-gram features that feed Linear SVM and LightGBM classifiers for each head, providing a fast and dependable starting point while class weights or focal loss handle imbalance.

In the upgraded configuration, a DistilBERT or Sentence-BERT encoder shares representations across the two classifier heads and applies class weighting or focal loss to manage skew while improving accuracy and robustness.

\medskip

\section*{4. Problem, impact, and challenges}
\subsection*{4.1 Problem Statement}
The objective is to automatically map messy, free-form resumes to clear job titles and recruiting families.

\medskip

\subsection*{4.2 Real-World Impact}
First, high-volume funnels such as campus and new-grad intakes generate thousands of resumes within days, so a reliable classifier routes candidates to Software, Data/ML, Cloud/DevOps, or Product/Business Analyst tracks and trims manual triage time.

\medskip

Additionally, early-career resumes are short and stylistically uneven, so consistent predictions reduce reviewer fatigue and guard against missed signals such as projects or skills that could otherwise slip past subjective screens.

\medskip

Furthermore, token and skill-level rationales explain routing decisions—for example, ``Python, SQL, Power BI'' pointing to the Data Analyst family—so students receive faster feedback that guides coursework and project choices.

\medskip

Finally, family labels enable pipeline analytics that report on track health, skill coverage, and background diversity while keeping personally identifiable information scrubbed.

\medskip

\subsection*{4.3 Why It’s Challenging (and How Our Design Addresses It)}
The task is challenging because resumes are unstructured, ambiguous, and imbalanced, requiring deliberate architectural and data-handling choices.
\begin{enumerate}
  \item Unstructured noisy text from multi-column PDFs and mixed formatting demands robust normalization, section-aware concatenation, and a transformer upgrade once the baselines stabilize.
  \item Synonymy and near duplicates such as SWE versus SDE or Backend with seniority suffixes require alias normalization, tail collapse, and Top-3 evaluation to capture realistic ambiguity.
  \item Long-tail class imbalance with roles represented by fewer than 60 examples calls for class weights or focal loss, the collapse strategy, and macro-F1 reporting.
  \item Hierarchy consistency between title and family depends on the two-head model backed by mapping checks and conflict flags that prompt human review.
  \item Privacy and fairness guardrails rely on PII scrubbing, exclusion of protected features, and slice metrics based on neutral proxies like resume length and section presence.
\end{enumerate}

\medskip

\subsection*{4.4 Justification}
Starting from 36 observed titles, the pipeline normalizes aliases, merges seniority variants, and targets $K\approx25$--$30$ classes so that granularity remains useful while approaching at least 60 examples per class whenever feasible. This collapse improves evaluations and supports defensible handling of sparse categories.

\medskip

The family space naturally populates five to seven SOC major groups, and training or evaluation is restricted to the families present in the dataset while keeping the mapping extensible for future data additions. This approach retains alignment with SOC definitions without forcing empty targets.

\medskip

Hierarchical agreement is enforced by mapping each predicted title to its family and comparing the result with the family head; any disagreements trigger a conflict flag and default routing to the family prediction for safer handling of intern and new-grad workflows.

\medskip

\subsection*{4.5 Risk and Mitigation}
Potential risks such as residual PII leaking through conversions, alias drift as new titles appear, label noise from weak supervision, class imbalance collapsing minority roles, overfitting to synthetic resumes, and opaque predictions are mitigated through automated scrubbing with spot audits, periodic alias-table reviews driven by active learning alerts, human-in-the-loop inspections of small cohorts each sprint, class-weighted or focal losses with tail buckets monitored via macro-F1, held-out ``real-style'' test slices with periodic shadow evaluation on fresh uploads, and reviewer-facing rationale logs that highlight key skills.

\setlength{\parindent}{15pt}

\section*{5. Data sources and collection plan}
\begin{itemize}
  \item Primary dataset: Kaggle Resume Dataset by \href{https://www.kaggle.com/datasets/rayyankauchali0/resume-dataset?resource=download}{rayyankauchali0} (CC BY 4.0)
  \item Mixture of real and synthetic anonymized resumes; inspired by ResumeAtlas (HuggingFace) and datasetmaster/resumes
  \item Preprocessing: PDF/DOCX $\rightarrow$ text, remove PII, normalize titles
  \item Label construction: most recent experience title, alias normalization, map to SOC family
  \item Train/validation/test splits stratified by role and source type; held-out real-only test set
  \item Metadata: rows by source, class counts, section presence, token length distributions
\end{itemize}

\section*{6. Expected dataset size and example datapoints}
\textbf{Expected size:} 3,500+ resumes total; $\geq$3,200 usable after filtering. \\
Fine-grained titles: ~60 classes, Occupation families: 23 classes

\textbf{Three example records:} \\[2pt]
\begin{tabular}{p{0.62\linewidth} p{0.32\linewidth}}
\toprule
Text snippet (input) & Label(s) \\
\midrule
"...10 yrs software design \& development -- senior Java developer / tech lead at Synnex (2014-Now)... REST APIs, Spring Boot..." & Software Engineer / Computer \& Mathematical \\
"...enthusiastic Java developer (3 yrs) -- reduced app memory 30\%, startup time 70\%. Roles at DaCoderz \& Quantexx..." & Java Developer / Computer \& Mathematical \\
"...8 yrs delivering enterprise Java/J2EE solutions; Spring Boot, microservices, AWS; Sr Java Dev @ Fiserv (2021-Now)..." & Software Engineer / Computer \& Mathematical \\
\bottomrule
\end{tabular}

\section*{7. Proposed solution}
\textbf{High-level approach:} Resume $\rightarrow$ preprocessing $\rightarrow$ feature extraction $\rightarrow$ two-head classifier (title + family) $\rightarrow$ evaluation and interpretation

\textbf{Features / inputs:}
\begin{itemize}
  \item Text features: normalized text (Experience $\rightarrow$ Skills $\rightarrow$ Summary/Education), TF-IDF vectors, or embeddings (DistilBERT / Sentence-BERT)
  \item Structured features: section flags, token counts, average bullet length, skill dictionary hits (Python, SQL, AWS, Kubernetes, Terraform)
\end{itemize}

\textbf{Model candidates:}
\begin{itemize}
  \item Baseline: Linear SVM, LightGBM
  \item Advanced: DistilBERT / Sentence-BERT two-head multi-task classifier
  \item Handle class imbalance with class weights/focal loss; collapse rare titles into Other
\end{itemize}

\textbf{Evaluation plan:}
\begin{itemize}
  \item Job title: Top-1 / Top-3 accuracy, macro-F1, confusion analysis
  \item Occupation family: accuracy + hierarchical correctness
  \item Calibration: Brier score, reliability curves
  \item Robustness: real-only vs mixed test, short vs long resumes
  \item Fairness: slices based on resume length and section presence
\end{itemize}

\textbf{Interpretability:}
\begin{itemize}
  \item Baseline: SHAP on n-grams and skill features
  \item Transformer: token-level attention highlights + short rationale
\end{itemize}

\textbf{Libraries / tools:}
\begin{itemize}
  \item pandas, numpy, scikit-learn, LightGBM
  \item transformers, sentence-transformers, torch
  \item shap, spaCy, pdfminer.six, python-docx
\end{itemize}

\textbf{Related work / references (2--5 sources):}
\begin{itemize}
  \item Multi-class text classification with TF-IDF + SVM/GBDT
  \item BERT fine-tuning for document classification with multi-task heads
  \item Hierarchical evaluation (family-level correctness when title is ambiguous)
  \item Industry practices for resume/job normalization using skill dictionaries
\end{itemize}

\end{document}
