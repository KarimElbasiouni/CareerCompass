\documentclass[11pt]{article}

% Change "review" to "final" to generate the final (sometimes called camera-ready) version.
% Change to "preprint" to generate a non-anonymous version with page numbers.
\usepackage[]{acl}
\usepackage{times}
\usepackage{latexsym}

% For proper rendering and hyphenation of words containing Latin characters (including in bib files)
\usepackage[T1]{fontenc}
% For Vietnamese characters
% \usepackage[T5]{fontenc}
% See https://www.latex-project.org/help/documentation/encguide.pdf for other character sets

% This assumes your files are encoded as UTF8
\usepackage[utf8]{inputenc}
\usepackage{microtype}
\usepackage{inconsolata}
\usepackage{graphicx}

% If the title and author information does not fit in the area allocated, uncomment the following
%
%\setlength\titlebox{<dim>}
%
% and set <dim> to something 5cm or larger.

\title{Group 85 Progress Report:\\CareerCompass}


\author{Karim Elbasiouni, Imran Chowdhury, Rami Abu Sultan \\
  \texttt{\{elbasik, chowdi13, abusultr\}@mcmaster.ca} }

%\author{
%  \textbf{First Author\textsuperscript{1}},
%  \textbf{Second Author\textsuperscript{1,2}},
%  \textbf{Third T. Author\textsuperscript{1}},
%  \textbf{Fourth Author\textsuperscript{1}},
%\\
%  \textbf{Fifth Author\textsuperscript{1,2}},
%  \textbf{Sixth Author\textsuperscript{1}},
%  \textbf{Seventh Author\textsuperscript{1}},
%  \textbf{Eighth Author \textsuperscript{1,2,3,4}},
%\\
%  \textbf{Ninth Author\textsuperscript{1}},
%  \textbf{Tenth Author\textsuperscript{1}},
%  \textbf{Eleventh E. Author\textsuperscript{1,2,3,4,5}},
%  \textbf{Twelfth Author\textsuperscript{1}},
%\\
%  \textbf{Thirteenth Author\textsuperscript{3}},
%  \textbf{Fourteenth F. Author\textsuperscript{2,4}},
%  \textbf{Fifteenth Author\textsuperscript{1}},
%  \textbf{Sixteenth Author\textsuperscript{1}},
%\\
%  \textbf{Seventeenth S. Author\textsuperscript{4,5}},
%  \textbf{Eighteenth Author\textsuperscript{3,4}},
%  \textbf{Nineteenth N. Author\textsuperscript{2,5}},
%  \textbf{Twentieth Author\textsuperscript{1}}
%\\
%\\
%  \textsuperscript{1}Affiliation 1,
%  \textsuperscript{2}Affiliation 2,
%  \textsuperscript{3}Affiliation 3,
%  \textsuperscript{4}Affiliation 4,
%  \textsuperscript{5}Affiliation 5
%\\
%  \small{
%    \textbf{Correspondence:} \href{mailto:email@domain}{email@domain}
%  }
%}

\begin{document}
\maketitle
% \begin{abstract}
% \end{abstract}

\section{Introduction}

Job recruiters have an increased need for understanding how to better align job applicants skills and experience with the best available job posting. Manually reviewing unstructured resumes is slow, subjective, and is inefficient in a job market where thousands are applying for the same job opening. Our project reframes resume understanding as a resume-to-title classification task: ingest raw resumes, scrub PII, and map each document to a canonical job title and occupational family. The resulting labels can power dashboards that highlight dominant skill profiles and align job applicants with industry and company demands. Building on our proposal, we committed to fully automating data preparation and establishing a baseline model which will later be used for comparison to a more capable model.
\section{Related Work}

Our approach draws on both older and newer ideas in text classification. Earlier works such as \citet{Joachims1998SVM} showed that linear SVMs, when paired with TF-IDF features, could handle high-dimensional text very well. Around the same time, \citet{SaltonBuckley1988TermWeighting} explored how adding term weights could make document retrieval more effective by balancing how often a word appears in one document versus across many documents. Those findings directly influenced how we designed our TF-IDF + LinearSVC baseline, keeping it interpretable and fast enough to deploy for advising job recruiters.

In recent years, researchers have shifted toward models that capture meaning rather than just word frequency. A notable example is JobBERT introduced by \citet{Decorte2021JobBERT}, a transformer-based model designed to capture how job titles and skills are related. \citet{Liu2022Title2Vec} developed Title2Vec, which maps job titles into a shared numerical space so that similar roles can be grouped and compared more easily. Large pretrained models like BERT \citep{Devlin2019BERT} have proven effective at understanding general language and can be adapted to domains such as career analytics, where wording and phrasing vary across resumes.

These studies helped guide how we approached CareerCompass. The early works gave us a solid foundation to begin our baseline model, and the newer methods discovered in recent years pointed us towards experimenting with fine-tuned BERT models to capture the subtle phrasing differences that can occur and better link resume text with standardized job titles.

\section{Dataset}

We use the publicly available Kaggle Resume Dataset by Rayyan Kauchali (2020; CC BY 4.0) \citep{Kauchali2020Resume} consisting of a mix of real and synthetic/anonymized resumes intended for NLP research. Each record includes structured text sections such as Summary, Skills, Experience, and Education, plus a noisy title-like category field (Category). Raw files are placed under \texttt{data/raw/}, and all preprocessing is orchestrated by \texttt{build\_clean\_dataset.py}.

Following our data plan, we applied a lightweight PII safeguard by redacting emails and phone numbers with deterministic regular expressions, then concatenated the key sections into a unified text field (\texttt{text\_clean}) and normalized it to \texttt{text\_norm}. Rows missing all major sections are dropped. This pipeline is reproducible via \texttt{build\_clean\_dataset.py} and supporting utilities (\texttt{src/data\_filter.py} and \texttt{src/text\_processing.py}), which write the canonical processed corpus to \texttt{data/processed/resumes\_v1.parquet} (and a CSV fallback).

Label construction follows our proposal and current implementation: raw titles from \texttt{Category} are normalized via a curated alias map into canonical titles (\texttt{title\_raw}), and each canonical title is mapped to a coarse occupation family (\texttt{y\_family}) using a lookup table (e.g., "Computers / IT", "Business / Finance"). This family mapping is conceptually aligned with SOC major groups, and unmapped or rare titles are assigned to \texttt{Other} to mitigate class imbalance.

Each record is assigned a stable \texttt{resume\_id} to support joins and splits. The processed dataset, exported in Parquet and CSV, serves as the single source for downstream feature extraction (TF-IDF) and supervised learning for two tasks: fine-grained job-title classification and coarse occupation family classification.

\section{Features}

Describe any features you used for your model, or how your data was input to your model. Are you doing feature engineering or feature selection? Are you learning embeddings? Is it all part of one neural network? Refer to item 2. This may range from 0.25 pages to 0.5 pages.

\section{Implementation}

Describe your model and implementation here. Refer to item 4. This may take around a page.

\section{Results and Evaluation}

How are you evaluating your model? What results do you have so far? What are your baselines? Refer to item 5. This may take around 0.5 pages.

\section{Feedback and Plans}

Write about your plans for the remainder of the project. This should include a discussion of the feedback you received from your TA, and how you plan to improve your approach. Reflect on your implementation and areas for improvement. Refer to item 6. This may be around 0.5 pages.

\section{Template Notes}

You can remove this section or comment it out, as it only contains instructions for how to use this template. You may use subsections in your document as you find appropriate.

\subsection{Tables and figures}

See Table~\ref{citation-guide} for an example of a table and its caption.
See Figure~\ref{fig:experiments} for an example of a figure and its caption.


\begin{figure}[t]
  \includegraphics[width=\columnwidth]{example-image-golden}
  \caption{A figure with a caption that runs for more than one line.
    Example image is usually available through the \texttt{mwe} package
    without even mentioning it in the preamble.}
  \label{fig:experiments}
\end{figure}

\begin{figure*}[t]
  \includegraphics[width=0.48\linewidth]{example-image-a} \hfill
  \includegraphics[width=0.48\linewidth]{example-image-b}
  \caption {A minimal working example to demonstrate how to place
    two images side-by-side.}
\end{figure*}


\subsection{Citations}

\begin{table*}
  \centering
  \begin{tabular}{lll}
    \hline
    \textbf{Output}           & \textbf{natbib command} & \textbf{ACL only command} \\
    \hline
    \citep{Gusfield:97}       & \verb|\citep|           &                           \\
    \citealp{Gusfield:97}     & \verb|\citealp|         &                           \\
    \citet{Gusfield:97}       & \verb|\citet|           &                           \\
    \citeyearpar{Gusfield:97} & \verb|\citeyearpar|     &                           \\
    \citeposs{Gusfield:97}    &                         & \verb|\citeposs|          \\
    \hline
  \end{tabular}
  \caption{\label{citation-guide}
    Citation commands supported by the style file.
  }
\end{table*}

Table~\ref{citation-guide} shows the syntax supported by the style files.
We encourage you to use the natbib styles.
You can use the command \verb|\citet| (cite in text) to get ``author (year)'' citations, like this citation to a paper by \citet{Gusfield:97}.
You can use the command \verb|\citep| (cite in parentheses) to get ``(author, year)'' citations \citep{Gusfield:97}.
You can use the command \verb|\citealp| (alternative cite without parentheses) to get ``author, year'' citations, which is useful for using citations within parentheses (e.g. \citealp{Gusfield:97}).

\subsection{References}

% (Remove nocite since all items are cited in Sections 2–3)
% \nocite{Kauchali2020Resume,Joachims1998SVM,SaltonBuckley1988TermWeighting,Devlin2019BERT,Decorte2021JobBERT,Liu2022Title2Vec}


Many websites where you can find academic papers also allow you to export a bib file for citation or bib formatted entry. Copy this into the \texttt{custom.bib} and you will be able to cite the paper in the \LaTeX{}. You can remove the example entries.

\subsection{Equations}

An example equation is shown below:
\begin{equation}
  \label{eq:example}
  A = \pi r^2
\end{equation}

Labels for equation numbers, sections, subsections, figures and tables
are all defined with the \verb|\label{label}| command and cross references
to them are made with the \verb|\ref{label}| command.
This an example cross-reference to Equation~\ref{eq:example}. You can also write equations inline, like this: $A=\pi r^2$.


% \section*{Limitations}

\section*{Team Contributions}

Write in this section a few sentences describing the contributions of each team member. What did each member work on? Refer to item 7.

% Bibliography entries for the entire Anthology, followed by custom entries
%\bibliography{custom,anthology-overleaf-1,anthology-overleaf-2}

\begin{filecontents}{custom.bib}
@misc{Kauchali2020Resume,
  author = {Rayyan Kauchali},
  title = {Resume Dataset},
  year = {2020},
  howpublished = {Kaggle},
  note = {CC BY 4.0},
  url = {https://www.kaggle.com/rkauchali/resume-dataset}
}
\end{filecontents}

% Custom bibliography entries only
\bibliography{custom}

% \appendix

% \section{Example Appendix}
% \label{sec:appendix}

% This is an appendix.

\end{document}
